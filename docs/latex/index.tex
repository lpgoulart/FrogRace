Algorithm to simulate a race between 3 frogs, writen in C++ on Programmig Language 1 grade at the IT Bachelor degree -\/ U\+F\+RN

\subsection*{Information}


\begin{DoxyItemize}
\item Doxygen documentation \href{index.html}{\tt available, on link}
\item More information and how-\/tos \href{https://lpgoulart.wixsite.com/it-developer}{\tt available, soon, on lpgoulart.\+wixsite/it-\/developer}
\item Source code \href{https://github.com/lpgoulart/FrogRace}{\tt available on Git\+Hub}
\end{DoxyItemize}

\subsection*{Getting Help}


\begin{DoxyItemize}
\item Please report bugs on the \href{https://github.com/lpgoulart/FrogRace/issues}{\tt issue tracker}.
\end{DoxyItemize}

\subsection*{Installation}

To get the lastet release

get the source code from Github repository\+:


\begin{DoxyCode}
1 $ git clone https://github.com/lpgoulart/FrogRace.git
\end{DoxyCode}


simple, use the Makefile on root folder\+:


\begin{DoxyCode}
1 $ make
\end{DoxyCode}


Then you\textquotesingle{}ll be able to use the binary file on bin directory

\subsection*{Getting Started}

Run\+: \begin{DoxyVerb}cd bin
./exec
\end{DoxyVerb}


Example\+: \begin{DoxyVerb}Set the distance: 
>5
Insert the name of the Frog¹: 
>Leo
Insert the name of the Frog²: 
>Aleco
Insert the name of the Frog³: 
>Beco

Set...Ready...Go!!
Jump distancie: 2|2|1
Jumps: 1
Distance: 2|2|1

Jump distancie: 2|3|2
Jumps: 2
Distance: 4|5|3

Winner: Aleco
2 Jumps
distancie: 5

Want to run again? 
(1) Run again
(0) End race 
>...\end{DoxyVerb}
 